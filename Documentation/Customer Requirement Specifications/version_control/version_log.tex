\section{Document Change History}

\begin{center}
\small\textit{Note: This change history table was generated by Autoleaf AI under the supervision of the Technical Writer. Only the most significant changes are highlighted, check the readme for more detailed information.}

\vspace{0.5cm}

\begin{tabular}{|p{0.05\textwidth}|p{0.09\textwidth}|p{0.17\textwidth}|p{0.14\textwidth}|p{0.39\textwidth}|}
\hline
\textbf{Ver.} & \textbf{Date} & \textbf{Modified Areas} & \textbf{Changed By} & \textbf{Description of Changes} \\
\hline
2.2 & 2024-10-17 & Functional Reqs., Non-Functional Reqs., User Classes, Structure & Analyst Team &  Restructures the document to improve clarity and traceability by introducing sub-requirements linked to functional and non-functional requirements. \\
\hline
2.1 & 2024-10-10 & User Stories, Scope, Non-Functional Reqs., Overall Description & Analyst Team & Focuses on improving document navigability through internal linking and restructuring user stories for clarity. \\
\hline
2.0 & 2024-10-03 & User Stories, Constraints, Software Attributes, Performance Reqs. & Analyst Team & Refines user stories, clarifies system constraints, and introduces specific software system attributes to guide development. \\
\hline
1.1 & 2024-09-24 & Introduction, Overall Description, Specific Requirements & Analyst Team & Expands upon the initial structure, providing detailed descriptions of user roles, system functionalities, requirements, and constraints. \\
\hline
1.0 & 2024-09-19 &  Introduction, Overall Description, Specific Requirements, Supporting Info. & Analyst Team & Establishes the initial structure and content of the Requirements Specification. \\
\hline
\end{tabular}
\end{center}

\vspace{1cm} 
